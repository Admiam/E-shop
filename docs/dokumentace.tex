\documentclass[12pt, a4paper]{report}

\usepackage[utf8]{inputenc}
\usepackage[IL2]{fontenc}
\usepackage[czech]{babel}
\usepackage{enumitem}
\usepackage{tocloft}
\usepackage{multicol}
\usepackage{pdfpages}
\usepackage{float}
\usepackage{tabularx}
\usepackage{listings}
\usepackage{parskip}
\usepackage{graphicx}
\usepackage{amsmath}
\usepackage[hidelinks]{hyperref}
\usepackage[nottoc]{tocbibind}

\usepackage[
    left=30mm, 
    right=30mm, 
    top=40mm, 
    bottom=30mm,
]{geometry}

% Sazba obrázků
\graphicspath{{Images/}} % Při vkládání obrázků se bude prefixovat tato relativní cesta.

% Numbering style modification
\renewcommand\thesection{\arabic{section}}
\renewcommand\thefigure{\arabic{figure}}
\renewcommand\thetable{\arabic{table}}

\newcommand{\lawyertalk}{\tiny}

\begin{document}

% Titulní strana
\begin{titlepage}
    \centering      % Odtud do konce prostředí bude vše na středu,
    \Large          % velkými písmeny
    \sffamily       % a bezpatkovým písmem.

    \includegraphics[width=.7\textwidth]{fav}

    Semestrální práce z předmětu

    Webové aplikace
    
    \vspace{18mm}
    {\Huge\bfseries Online obchod T-space}

    \vspace{18mm}
    \today                          % Čas je získán ze systému.

    \vfill                          % Vyplní prostor
    \raggedright                    % Vše bude zarováno do leva.
    \textsl{\lawyertalk Autor:}\\   % Vtípek z přednášky + ukázka tvorby makra a přidání sémantiky do stylu textu.
    Adam Míka\\               % Příkaz \\ provede násilný zlom řádky.
    A22B0319P\\
    \texttt{mikaa@students.zcu.cz}
    
    \vspace{\baselineskip}
    \textsl{Cvičící:}\\
    Ing. Michal Nykl Ph.D.\\
    \texttt{nyklm@ntis.zcu.cz}
\end{titlepage}

% obsah
\tableofcontents

\pagebreak

\section{Zadání}
\subsection{Nutné požadavky na všechny samostatné práce}

\begin{enumerate}[left=1cm] % Nastavení odsazení pro všechny položky v seznamu
\item Technologie - povinně HTML5, CSS, PHP a SQL (MySQL nebo jiná databáze), volitelně šablony, JavaScript, AJAX, Bootstrap apod.
\item Aplikace musí dodržovat MVC architekturu a využívat OOP (min. controllery a model).
\item Web má jeden vstupní soubor (obvykle index.php), který na základě parametrů URL adresy provede požadovanou akci (tj. zavolá příslušný controller) a vypíše výstup uživateli.
\item Pro práci s databází musí být využito PDO nebo jeho ekvivalent.
\item Web musí být chráněn proti útokům typu XSS a SQL Injection.
\item V databázi musí být hesla hashována.
\item Web musí využívat upload souborů.
\item Web musí mít responzivní design (alespoň pro PC a mobil).
\item Web musí mít alespoň 3 uživatelské role (po přihlášení v systému provádí příslušné činnosti, např. autor, recenzent, admin).
\item K aplikaci musí být dodána dokumentace (viz dále) a skripty pro instalaci databáze (např. získané exportem databáze).
\item Práce musí být osobně předvedena cvičícímu a po schválení odevzdána na CourseWare či Portál.
\item Aplikaci není možné realizovat s využitím ucelených PHP frameworků (zakázáno např. Nette, Symfony atd.). Použití jejich komponent je možné pouze po schválení vyučujícím.
\item Pro front-end je vhodné využít framework Bootstrap (getbootstrap.com) nebo jeho ekvivalent.


\end{enumerate}


\subsection{Dokumentace}
\begin{enumerate}[left=1cm] % Nastavení odsazení pro všechny položky v seznamu
\item Vaše jméno, email, datum vytvoření, název předmětu a název aplikace/tématu.
\item URL vytvořených stránek (pokud jsou zveřejněny na serveru students.kiv.zcu.cz či jinde)
\item Popis použitých technologií - uveďte hlavně, ve které části jste kterou technologii použili.
\item Popis adresářové struktury aplikace - co je ve kterých adresářích a souborech.
\item Popis architektury aplikace - co mají na starosti které třídy (popř. lze využít i UML diagramy).
\item Seznam defaultních uživatelů včetně loginů a hesel.
\item U alternativního zadání uveďte celé, cvičícím schválené zadání práce.
\item Dokumentaci netiskněte, ale odevzdejte ji ve formátu PDF spolu s aplikací.
\end{enumerate}



\subsection{Hodnocení samostatné práce a získání zápočtu}
\begin{enumerate}[left=1cm] % Nastavení odsazení pro všechny položky v seznamu
    \item Hodnocení práce je rozděleno na povinné požadavky a volitelná rozšíření, přičemž detailní podmínky sdělí a vysvětlí každý vyučující na prvním cvičení.
    \item Pro cvičení, která vede M.Nykl, je hodnocení uvedeno v záložce menu Hodnocení SP (Nykl).
    \item Pro získání zápočtu je nezbytné splnit nutné požadavky a získat ze semestrální práce minimálně 20 bodů (max. 60).
\end{enumerate}
Dodatečné poznámky k hodnocení:
\begin{enumerate}[left=1cm] % Nastavení odsazení pro všechny položky v seznamu
    \item Zákaz plagiátorství - práci musí student vytvořit samostatně. Nelze ji tedy zkopírovat např. z Githubu a ani nelze "udat" starou práci ze střední školy (obvykle nesplňuje nutné požadavky). Plagiátorství je odhalováno automatizovaným systémem a hodnoceno odebráním zápočtu. 
    \item Odevzdání práce v průběhu cvičení v ZS (tzv. do Vánoc) je ohodnoceno bonusovými body.
    \item Bootstrap či ekvivalent - obvykle umožňují snadno zajistit responzivitu a pěkný design (dohromady hodně bodů).
    \item Github, Bitbucket  - práce je uložena v repozitáři (snadné body; výborná záloha historie kódu).

\end{enumerate}

\subsection{Předvedení a odevzdání SP}
\begin{enumerate}[left=1cm] % Nastavení odsazení pro všechny položky v seznamu
\item Nutnou součástí odevzdání je osobní předvedení práce vyučujícímu, které je možné pouze na cvičeních nebo ve stanovených termínech (cca 3 za zkouškové období).
\item Vytvořené webové stránky při předvádění cvičícímu poběží na počítači, ke kterému bude mít cvičící fyzický přístup (např. počítač v učebně nebo notebook, který máte s sebou). Webovým serverem bude buď localhost (127.0.0.1), nebo univerzitní server students.kiv.zcu.cz.
\item Akceptovaná aplikace (tj. zdrojové kódy včetně dokumentace ve formátu PDF) bude odevzdána do odevzdávacího portletu na Portal či Courseware. Archiv bude pojmenovaný PRIJMENI-JMENO.ZIP a v nejvyšší úrovni adresářové struktury bude soubor readme.txt s popisem obsahu jednotlivých adresářů. Práci ale odevzdejte až po jejím akceptování vyučujícím.
\end{enumerate}


\section{Popis použitých technologií}
\subsection{JavaScript}
JavaScript je skriptovací programovací jazyk, který umožňuje interaktivitu na stránkách. Používá se pro manipulaci s obsahem HTML stránek, zpracování událostí a komunikaci s webovým serverem.

\begin{lstlisting}
    function limitT(a) {
        if (a.length > 32) return a.substring(0, 90)+"...";
    }
    $("#content p").text(function() {
        return limitT(this.innerHTML)
    });
\end{lstlisting}
JavaScript je použit pro klientovou validaci vstupních formulářů, interaktivní prvky na stránce a reakci na uživatelské události.
Například pro omezení počtu znaků v popisu produktu.

\subsection{jQuery}
% \includegraphics[width=1\textwidth]{druhy.png}
jQuery je rychlá, malá a bohatá knihovna pro JavaScript, která zjednodušuje manipulaci s dokumentem HTML, obsluhu událostí, animace a práci s AJAXem.

jQuery byl použit pro usnadnění a zjednodušení některých operací v JavaScriptu, jako například selekce elementů, manipulace s DOM a práce s AJAXem.

\subsection{MVC Architektura}
MVC (Model-View-Controller) je architektonický vzor, který rozděluje webovou aplikaci do tří základních komponent - modelu, pohledu a řadiče. To pomáhá oddělit logiku aplikace od prezentační vrstvy a usnadňuje údržbu a rozšiřitelnost.

MVC architektura byla použita pro organizaci a strukturování kódu semestrální práce. Třída WhiteTeaController slouží jako řadič, který ovládá logiku pro stránku O nás, MyDatabase zastává roli modelu pro práci s databází a pohled je oddělen do samostatného souboru whitetea.php.





\section{Popis adresářové struktury aplikace}

\begin{enumerate}[left=1cm]
\item assets: Obsahuje statické soubory, jako CSS styly (style.css) a JavaScriptové skripty (script.js), které jsou používány ve webové aplikaci.
\item controllers: Obsahuje kontroléry aplikace. IController.interface.php může být rozhraním, které implementují všechny kontroléry, a whiteTeaController.php je konkrétní implementací kontroléru pro stránku O nás.
\item models: Obsahuje modely aplikace. MyDatabase.class.php může obsahovat třídu pro práci s databází a userManage.php může zahrnovat třídu pro správu uživatelů.
\item views: Obsahuje šablony pohledů aplikace. whitetea.php může obsahovat HTML a PHP kód pro zobrazení stránky O nás.
\item index.php: Hlavní vstupní bod webové aplikace. Zde může být definována logika pro směrování a spouštění aplikace.

\end{enumerate} 

\section{Popis architektury aplikace}
WhiteTeaController
\begin{enumerate}[left=1cm]
    \item Zajišťuje inicializaci a správu objektů pro práci s databází a uživateli.
    \item Obsahuje metodu show, která zpracovává požadavky na zobrazení stránky O nás, manipuluje s daty a výsledek vrací ve formě řetězce HTML.
\end{enumerate} 

MyDatabase
\begin{enumerate}[left=1cm]
    \item getTea: Získává data o čaji podle zadané kategorie.
    \item addToCart: Přidává produkt do košíku uživatele.
    \item decrease: Sníží množství produktu v databázi.
    \item updateQuantity: Aktualizuje množství produktu v databázi.
\end{enumerate} 

UserManage
\begin{enumerate}[left=1cm]
    \item userLogout: Odhlašuje uživatele.
    \item isUserLogged: Kontroluje, zda je uživatel přihlášen.
    \item getLoggedUserData: Získává data o přihlášeném uživateli.
\end{enumerate}

Popis vzájemných vztahů:
\begin{enumerate}[left=1cm]
    \item WhiteTeaController má vlastní instance tříd MyDatabase a UserManage, které využívá pro práci s databází a správu uživatelů.
    \item MyDatabase a UserManage jsou samostatné modelové třídy pro práci s databází a uživatelskými účty.
    \item WhiteTeaController komunikuje s MyDatabase a UserManage prostřednictvím jejich veřejných metod.
    \item WhiteTeaController je součástí architektury MVC, kde plní roli řadiče (Controller), MyDatabase je model pro práci s daty, a UserManage je další model pro správu uživatelů.
\end{enumerate}


\section{Seznam defaultních uživatelů}

Administrátor
\begin{enumerate}[left=1cm]
    \item Login: admiam
    \item Heslo: adam
\end{enumerate}

Admin
\begin{enumerate}[left=1cm]
    \item Login: rychlikj
    \item Heslo: heslo12345
\end{enumerate}

Skladník
\begin{enumerate}[left=1cm]
    \item Login: jensísek
    \item Heslo: heslo12345
\end{enumerate}

Zákazník
\begin{enumerate}[left=1cm]
    \item Login: martinjekokos
    \item Heslo: MaritinMaritin
\end{enumerate}

\section{Závěr}
Veškerá architektura, adresářová struktura a implementace tříd webové aplikace byla navržena s ohledem na efektivitu, modularitu a snadnou údržbu. Aplikace využívá technologie jako JavaScript, jQuery a MVC architekturu k dosažení co nejlepší uživatelské zkušenosti a jednoduché správy kódu.

Adresářová struktura byla pečlivě organizována, s oddělením statických souborů, kontrolérů, modelů, pohledů a konfiguračních souborů. Tím je dosaženo čistého oddělení zodpovědností a zvyšuje se přehlednost projektu.

MVC architektura, kterou aplikace následuje, zajišťuje, že logika, data a prezentační vrstva jsou odděleny, což usnadňuje úpravy, rozšíření a testování jednotlivých částí aplikace.

Celkově lze konstatovat, že navržená webová aplikace splňuje požadavky na strukturu, organizaci a efektivitu. Využívání moderních technologií a návrhových vzorů přispívá k optimálnímu vývoji a udržení kvalitního kódu. Projekt je připraven k dalšímu rozvoji, ať už se jedná o přidání nových funkcí, optimalizaci či rozšíření.

\end{document}
